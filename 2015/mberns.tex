\documentclass[aspectratio=169]{beamer}
\usepackage[utf8]{inputenc}
\usepackage[T1]{fontenc}
\usepackage{listings}
\title[Developing Java ATK Wrapper]{Developing Java ATK Wrapper}
\author{Magdalen Berns \ \texttt{email:m.berns@thismagpie.com}}
\date{August 7, 2015}

\usetheme{guadec}

\AtBeginSection{\frame{\sectionpage}}

\begin{document}

\begin{frame}[containsverbatim]
\titlepage
\end{frame}

\begin{frame}[containsverbatim]
\frametitle{What is Java ATK Wrapper?}
 \begin{center}
Orca screenreader users rely on Orca's voice to navigate the desktop. Orca needs GNOME's ATK to allow it to speak GTK app widgets (and windows) to the user.
\end{center}

\begin{center}
When an Orca user is reliant on a Java app, the java-atk-wrapper acts as a bridge between AWT events from Swing widgets (and windows) and ATK.
\end{center}

\begin{center}
A company called Sun contributed the Wrapper to GNOME in 2009. It was mainly developed by Ke Wang until 2011; between then and 2014, the wrapper was not actively maintained then Alejandro and I did some work to fix the build and I became maintainer; so, the wrapper was pretty broken (and a little outdated) and still needed a lot of work to make it into a more complete package.
\end{center}
\end{frame}

\begin{frame}[containsverbatim]
\frametitle{How is the java wrapper organised?}
The wrapper comprises of a Java library and a C library organised into two separate subfolders:
    \begin{itemize}
        \item wrapper
            \begin{itemize}
        		\item Listens to AWT events.
		\item Adapts JAAPI to ATK interfaces.
	    \end{itemize}
        \item  jni
       \begin{itemize}
        	\item Converts AWT events into GSignals.
	\item Implements ATK class and interface pointers by calling native java functions in C.
    \end{itemize}
    \end{itemize}
\end{frame}

\begin{frame}
\frametitle{What needs done to the java-atk-wrapper module in GSoC?}
      \begin{center}
      This project is mainly about implementing ATK class and interface pointers which are missing from the wrapper's library but it's also about fixing many bugs.
      \end{center}
            \begin{center}
      Working with ATK, JNI, GLib, GTK, JAAPI and being a module maintainer has its learning curves; implementing the pointer functions with JNI helps me learn the libraries better.
      \end{center}
                  \begin{center}
      A number of crashes and more trivial bugs have now been fixed, the build has been improved and since most of the functions have now been implemented, I have started looking into better reproducable tests and thinking about converting more kinds of AWT events into GSignals with JNI, since some of these are also still missing.
                  \end{center}
\end{frame}

\begin{frame}
\frametitle{Thank You}
\begin{center}
  Thanks for listening and please feel free to check out the java-atk-wrapper on git and get involved!
\end{center}
\end{frame}

\end{document}
